\subsection{汉莎同盟}

汉莎同盟是历史上德意志北部城市之间形成的商业、政治联盟。汉莎一词,德文意为“公所”或者“会馆”,来自哥特语“军队”或“ 连队”,初指同业公会或行会,后指在国外的德国商人团体。在14世纪以前,“汉莎”只是用来称呼伦敦和布鲁日的德意志人团体,如“科隆汉莎”、“汉堡汉莎”,至此也被用来统称北德意志商人的商业共同体,商人汉莎自此演变为城市汉莎。


\subsubsection{起源}

汉莎同盟的起源可以追溯至12世纪。科隆、吕贝克、汉堡和不来梅由于地理位置的便利,这四座城市先后继承了诺曼征服以前丹麦、挪威、冰岛与盎格鲁-撒克逊诸国和爱尔兰之间的贸易关系。科隆商人在英格兰取得了贸易特权和建立侨居地的特权,而且曾慷慨地为狮心王理查支付赎金,以此获得了在英格兰的免税特许状。此后继位的英格兰国王相继给予吕贝克和汉堡的商人以相似的特权,这几个城市的商人在英格兰形成了早期的商业联盟。1158年/1159年萨克森公爵狮子亨利从荷尔斯泰因公爵手中夺占吕贝克城。通过这一占领,经济繁荣的萨克森王国打开了通往波罗的海的通道。下德意志地区的商人很快以此为基地,将商业活动扩展至整个波罗的海地区。吕贝克商人很快在瑞典城市维斯比和俄罗斯的赫尔姆加德(诺夫哥罗德)设立了商站,并在1299年控制了里加的商业。  


1293年,应吕贝克的一再要求,来自梅克伦堡和波美拉尼亚的商人在罗斯托克召开代表大会,决定今后凡与他们有关的案件都按吕贝克法律解决。有26个城市投票通过这一决议。吕贝克成为汉莎同盟总部所在地,吕贝克法律成为同盟之共同法律。汉莎同盟成立之后,其宗旨很快从镇压海盗、打击掠夺行为和取消不合理的通行税等保护商业的行为转至在海外扩展商业势力,垄断商业利益,压制来自同盟外的德意志城市以及来自英国、俄罗斯和佛兰德本土的商业竞争。随着北方来自丹麦和南方来自勃兰登堡的军事威胁的扩大,汉莎同盟开始建造海军,并由单纯的经济同盟演化成经济-政治-军事同盟。许多城市为了获得商业利益或获得安全保障而加入同盟。到14世纪中叶,汉莎同盟已经扩展至波罗的海南岸、东岸的所有德国港口城市,并扩展到英国、佛兰德、丹麦、斯堪的纳维亚、俄罗斯、芬兰等地。波罗的海东岸的德意志殖民者按照严格的汉莎法律建造了一大批汉莎城市。13世纪逐渐形成,14世纪达到兴盛,加盟城市最多达到160个。1367年成立以吕贝克城为首的领导机构,有汉堡、科隆、不来梅等大城市的富商、贵族参加。拥有武装和金库。1370年战胜丹麦,订立《斯特拉尔松德条约》。同盟垄断波罗的海地区贸易,并在西起伦敦,东至诺夫哥罗德的沿海地区建立商站,实力雄厚。15世纪转衰,1669年解体。


\subsubsection{汉莎同盟巅峰时期}
到14世纪中叶,汉萨同盟已经扩展至波罗的海南岸、东岸的所有德国港口城市,并扩展到英国、佛兰德、丹麦、斯堪的纳维亚、俄罗斯、芬兰等地。波罗的海东岸的德意志殖民者按照严格的汉萨法律建造了一大批汉萨城市。

随着汉萨商人的活动,北欧和西欧君主国在与汉萨同盟建立商业关系后获得了很大的商业利益。这些国家的经济大多建立在庄园制基础上,产品单一,而且生产效率低下。汉萨同盟进入后,其本国生产的原料可以以更高的价格在外国市场上出售,同时可以获得质量更优良的外国进口产品。汉萨商人的进出口税又为国库提供了稳定而可靠的收入来源,因此各国纷纷欢迎汉萨同盟在本国开设商站。这些商站集市场、商人住宅和办公室、仲裁所、作坊、工场、库房为一体,并设有防御设施,昼夜有人看守,以防当地人袭击。早期的主要商站设立在瑞典哥得兰岛上的维斯比,因丹麦袭击该地,曾引发汉萨同盟与丹麦的战争。随后在英国的伦敦、佛兰德的布鲁日、挪威的卑尔根和俄罗斯的诺夫哥罗德出现了四大汉萨同盟商站。

伦敦商站又被称为“钢院商站”(en:Steelyard/de:Stahlhof)(该名起源有争议,一说是源自德文“货栈”一词Stapelhof的讹传,还有说法认为此处出售来自德国的钢锭和钢坯),设立于1282年,由高墙围护,有自己的码头,主要建筑为一座三层楼房,还有存放文件的塔楼,以及庭园和武器库。汉萨同盟在各商站驻有商业代表,由声誉良好的已婚商人担任。妇女严禁进入商站。除了四大商站外,汉萨同盟还在其他一些外国城市(如英国的羊毛出产地)设有营业所和账房。

14世纪晚期和15世纪早期是汉萨同盟的鼎盛时期,虽然其商站只局限于波罗的海、北海和俄罗斯,但是其商船却远及法国、西班牙和葡萄牙,从南欧运来橄榄油、水果、各种酒和盐。其注册商船大至1000到2000吨,汉萨商人通过向神圣罗马帝国的封建领主和其他国家的君主提供借款而获得了一项又一项的特许权。在其全盛时期,汉萨同盟左右着丹麦和瑞典的王位继承人人选,而英国国王甚至不止一次将王冠抵押给汉萨商人换取贷款,或向其央借舰队和海员。汉萨同盟积累了大量财富,并凭借商业和武力赢得了各国的。
  
  
\subsubsection{汉萨同盟下的个人主义}

汉萨同盟的起源在于科隆商人为了自己的贸易特权为英格兰提供的资金支持,由此获得机会的德意志商人们为了自己的商业利益,积极向波罗的海地区扩展商业活动。汉萨同盟的前身由此产生。从此以吕贝克为中心,四周的城市为了自己的商业政治利益,纷纷加入同盟之内。它们有的是为了追求自身利益的最大化,有的是为了打压其他城市的贸易利益,还有的是为了对抗来自国内的压力,而有的是个人为了自己的利益使得整个城市加入同盟。汉萨同盟无论从它的兴起还是达它的鼎盛时期,都充斥着浓厚的利己主义和个人主义色彩。其中无论是少数人为了自己的利益缔结契约,还是个别城市为了自己的特权对抗中央,打压其他城市的行为,都是个体以自己为考虑对象,向群体施加影响的典型个人主义实例。但是这种个人主义即有其优势又有其弊端,在个人主义影响下,产生了无数的历史事件,一桩又一桩经典的契约精神实例,使得整个欧洲的经济贸易迅速发展,同时也成就了一个又一个的时代潮流下的个人。也正是这种个人主义行为,强烈的利己主义心态,使得汉萨同盟能够在百年之内迅速掀起,而又在利益存在矛盾是又迅速衰弱。

%\endinput  
 
\subsection{中国古代陆上丝绸之路}

19世纪末,德国地质地理学家李希霍芬在《中国》一书中,把“从公元前114年至公元127年间,中国与中亚、中国与印度间以丝绸贸易为媒介的这条西域交通道路”命名为“丝绸之路”,这一名词很快被学术界和大众所接受,并正式运用。其后,德国历史学家郝尔曼在20世纪初出版的《中国与叙利亚之间的古代丝绸之路》一书中,根据新发现的文物考古资料,进一步把丝绸之路延伸到地中海西岸和小亚细亚,确定了丝绸之路的基本内涵,即它是中国古代经过中亚通往南亚、西亚以及欧洲、北非的陆上贸易交往的通道。

传统的丝绸之路,起自中国古代都城长安,经中亚国家、阿富汗、伊朗、伊拉克、叙利亚等而达地中海,以罗马为终点,全长6440公里。这条路被认为是连结亚欧大陆的古代东西方文明的交汇之路,而丝绸则是最具代表性的货物。数千年来,游牧民族或部落、商人、教徒、外交家、士兵和学术考察者沿着丝绸之路四处活动。

随着时代发展,丝绸之路成为古代中国与西方所有政治经济文化往来通道的统称。有西汉张骞开通西域的官方通道“西北丝绸之路”;有北向蒙古高原,再西行天山北麓进入中亚的“草原丝绸之路”;有长安到成都再到印度的山道崎岖的“西南丝绸之路”;还有从广州、泉州、杭州、扬州等沿海城市出发,从南洋到阿拉伯海,甚至远达非洲东海岸的海上贸易的“海上丝绸之路”等。
  
\subsubsection{中国丝绸之路下集体主义:}

丝绸之路的典型代表人物就是汉朝的张骞,他作为国家的代表出使西方国家,在遭到敌对国家扣留后,仍然坚持向有关国家传递汉朝希望与他们发展贸易,互惠互利的想法。在敌对国家重金收买,高官厚禄的诱惑下,张骞不以个人利益为优先,而是以汉朝和西方国家的利益为重。在被扣留20余年后,张骞在汉朝的外交斡旋下回到自己的国家,此后仍然为汉朝与西方国家的集体利益而工作。这种先以多数人的利益为优先后考虑自己利益的做法就是中国丝绸之路的一项强大思想精神。先天下之忧而忧 ,后天下之乐而乐的集体思想,不但体现在丝绸之路的个人身上,同时还体现在丝绸之路的起源国家中国身上。当时的中国汉朝,往往为了维护丝绸之路沿途国家的利益,向他们单方面提供经济以及货物,技术的支持。而这种行为有时是损害自身利益的,比如限于当时的交通技术,运送货物的成本往往高于货物本身,而考虑到维护周边国家的经济利益,与两国的友好关系,汉朝仍然会向贸易国运送货物。
这种以集体为单位,不计个体损失照顾群体的行为,是中国的一种集体文化。并一直在中国传承,如明朝的郑和下西洋,和现在的中国国家战略“一带一路”。
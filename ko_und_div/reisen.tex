\subsection{中国人旅游}

随着旅游业的日渐兴起,无论过不过节,在世界各地,总能看到中国人的旅游大军,分布在世界的各个角落,网上关于中国旅游的话题也是吐槽多多。那么中国人和外国人的旅游到底有什么区别呢?

中国人旅游喜欢扎堆,特别是在节假日出游,到处都是人挤人,尤其在比较有名的旅游景点,你看到的到处都是乌央乌央的人头,毫无放松旅游的感觉。就算是在国外,比如在泰国游玩拍个照的时候,看到的都是一半人和一半景。但是外敷偶人旅游,同样是在泰国,却可能完全不一样,他们大都似电影里演的那样,很休闲的躺在沙滩椅上,晒着日光浴,而旁边还放一杯鸡尾酒。

扎起中国,许多大叔大妈,爷爷奶奶旅游时都强调低价,什么叫做低价。比如说,“999”玩遍泰国,“598港澳游,全程星级酒店”的这类的低价团,都备受追捧。但实际上就算是很正规的旅游团都会存在强迫购物的项目,更不要提这种明眼人一看便知的购物团了。不过在国外旅游,完全没有低价团这件事,往往都需要付足了各种费用,在泰国和越南这种收小费的习惯,其实就是他们养出来的。

中国人旅游爱逛商业街,国内外各地景色不同,但在中国旅游人的眼中,不到商业街就不算是来到了哪里旅游,尤其是以稍老一辈人最为明显,他们不论去哪儿旅游,往往都会在当地的商场里逛一圈,不论是衣服也好,纪念品也好,都很感兴趣。

中国人旅游在国外人看来,都是不可思议的,在国内人看来也是啼笑皆非。不过,这样的旅游方式其实也是非常有意思的。我觉得,作为年轻人,年轻的时候,在选择旅游方式时可以追求点个性。但是等步入了老年,很多人一起热闹旅游也未必不是好事。

近年来,越来越多的中国人都开始讨厌“扎堆”旅游,假日前夕,总有不少人在网上抱怨未来的人流高峰,甚至发誓绝不出门。但实际上,每逢节假日,国内各个主要景点依旧人山人海,人流量永远有增无减,和网上的舆论形成鲜明对比。为什么都抱怨人多,但中国人还是喜欢到人多的地方旅游呢?这是因为从全局考虑,人多也有人多的好处。
\begin{figure}[htb]
    \centering
    \includegraphics[width=0.6\linewidth]{zgrly}
    \caption{中国人旅游}
\end{figure}





\begin{enumerate}
    
\item \textbf{人多的地方更安全}

许多年轻的旅行者,为了规避人流高峰,喜欢挑选一些人迹罕至的地方去旅游。而人迹罕至,也就意味着这个地方充满了不可知的危险,一旦遇险,你连求援的对象都没有,只能自求多福。而人多的地方,恰恰没有这样的风险。一般来讲,热门景点都已经遭到了完全的开发,排除了大部分未知的危险因素。即便你在这些地方遭到困境,也随时可以向管理人员或身边的游客求助。过去,网上经常有驴友登山失联的消息。试想一下,如果这些山上面到处都是游客,你还会担心失联吗?事实上,在人多的地方,你想要失联都失联不了,因为没走几步,肯定会被人群挤回来。你可能会说,人多难道不是会有踩踏的危险吗?没错,但从统计数据来看,人多地方发生踩踏的概率,远远低于在荒郊野岭被野兽吃掉的概率。再比如说你的手机没电了,在人多的地方就可以很方便的借到充电宝,也是非常幸福的呀。
\begin{figure}[htb]
    \centering
    \includegraphics[width=0.6\linewidth]{zgrly2}
    \caption{长城旅游}
\end{figure}





\item \textbf{人多的地方更温暖}

通常情况下,旅游景点的温度,都低于我们平常生活环境的温度。特别是国庆这样的节日,刚好遇到秋季,气温急剧转凉,大家从温暖的室内出来,奔赴山水之间,最大的感受就是真TMD冷。这个时候,如果你去的碰巧是荒无人烟的地方,那么很可能夜里会在帐篷里冻死。 相反,如果你去的是人流密集的场所,几万人身上散发的热量,很快就会把整个地方熏得热火朝天。

\begin{figure}[htb]
    \centering
    \includegraphics[width=0.6\linewidth]{zgrly3}
    \caption{人多暖和}
\end{figure}



\item \textbf{人多的地方更有趣}

你去那些人群稀少的地方旅游,乍看好像无人打扰,自由自在,但很快你就会陷入空虚和无聊之中。事实上,风景永远是千遍一律的,珠峰上的景色再壮丽,让你在上面独居三十天,是个人都会崩溃。但有一种东西,永远花样翻新,永远不会腻味。没错,那就是人。正所谓“有人的地方就有江湖。”有多少个人,就有多少个江湖。而人多的经典,就好像一个浩瀚的武侠世界。在那些地方,你不仅可以看到美景,而且可以看到身边频频上演的江湖大戏,像情侣吵架、兄弟相争之类的剧目,在景点里随处可见。当然现在最流行的剧目,则是低价旅行团里导游和团客的斗争,揉和了爱情、动作、政治、商战和反乌托邦元素,倍受观众欢迎。去人多的地方旅行,“既是自然之旅,也是人文之旅。”
\begin{figure}[htb]
    \centering
    \includegraphics[width=0.6\linewidth]{zgrly4}
    \caption{人多有趣}
\end{figure}


\item \textbf{主要是从众心理作崽}

中国人素来有凑热闹的传统,只是名气大的地方总想也去打卡一番,即使回来不停吐槽也在所不惜。这类游客通常只是喜欢出去玩,但对旅行的认知并没有到很深的层面,除了被其他人走烂的景点,别的地方都没有什么印象。
\end{enumerate}

\subsection{德国人不同于中国人的旅游方式}

生活不止眼前的苟且,还有诗和远方--- 德国人深谙这个道理,在忙碌的工作后安排一场放松身心的旅行对他们来说必不可少。2015年,德国人的度假天数总计高达8.7亿天,由此可知这是一个个酷爱度假的民族。

\begin{figure}
    \centering
    \includegraphics[width=0.6\linewidth]{de_reisen1}
    \caption{德国旅游}
\end{figure}
\paragraph{假日来历:}

“旅游”在全欧或全球范围内都意义重大。对德国人来说,度假可以算最重要的事情之一。人们可以通过度假忘记工作日的疲惫,重新充满能量。“Urlaub(假日)”一词源于中古高地德语“Urloup”。这个词在当时表示上级对某人离开工作岗位的许可。

自1903年起,德国通过带薪休假规定确定了休假天数。那时所有员工都享有三日带薪年休假的权利。早在1841年,德国就出现了由旅游公司Thomas Cook提供的第一个全包旅行。旅客们乘坐火车,从莱斯特到拉夫堡,享受着价格内包含的餐食和乐队演奏等服务。

现在,度假对德国人来说变得更重要了。2013年,全球旅客数量达10.9亿,较2012年增长5\%。旅游行业也得到极大地发展,仅在德国就有300万人从事该行业。



\paragraph{德国假日天数、游客特征:}

现在,德国以年均41天假日和法国在欧洲范围内并列第一,因此德国人和法国人可谓是欧洲的“度假冠军”。就德国而言,除了由雇主保障的30天假日外,在许多联邦州还有额外11天的法定节假日,所以每年德国人常有41天时间供自己随意使用。与之相反,比利时人以年均仅20天带薪假日以及9天法定假日排名垫底。

当德国人出发去度假时,他们当然会做好万全准备。通常来说他们都会带些什么呢?通过什么人们可以一眼就认出德国游客呢?如果德国人准备享受一次火热的夏日风情度假,那么他们首先必备的就是 --- 一顶帽子,帽子可以保护脸部不受阳光暴晒。另外,一个典型德国人在度假时会在脖子上挂一个相机,以便记录下重要的景点风光。穿着方面,他们通常会穿一件衬衫和一条七分裤,必不可少的当然还有一双凉鞋搭配白色袜子。

\paragraph{最受欢迎目的地、住宿方式:}

虽然这有些不可思议,但相比于去别的国家旅游,德国人确实更喜欢留在德国。2015年,像以前一样,德国以29\%这一相当高的比例荣登最受国人喜爱的旅行目的地榜首。2014年,这一比例甚至达到31\%。除此之外德国人也喜欢在欧洲范围内旅行。比如:西班牙以13\%的比例成为最受德国人青睐的国外游目的地,意大利以8.2\%的比例紧随其后。其余较受欢迎的还有奥地利、克罗地亚、希腊和法国等地。

在人们享受应得的假期前,当然得提前订好住宿。那么德国人最喜欢哪种住宿方式呢?对大多数德国人来说,舒服是住宿的第一要义,只有这样他们才能尽可能好地放松,忘记工作日的紧张和压力。另外人们也希望在度假过程中享受到无微不至的服务。所以48\%的德国人选择酒店或舒适的旅馆作为旅行中的过夜地。度假屋以23\%的比例屈居第二。相较起来,帐篷倍受冷落,2015年,只有6\%的德国人选择了此种过夜方式。



\paragraph{度假出行方式:}

在多种不同的出行方式中,开车出行最受德国人青睐,首要原因还在于他们最爱在本国内旅行。因此约有45\%的德国人都喜欢将汽车作为旅行的交通工具。但飞行正行驶在“超车道”上。2015年,约有40\%的德国人乘坐飞机出游。对于前往国外旅游的人,这一比例甚至更高,达到56\%。德国人也会使用公车和铁路交通,不过其占比仅有7\%和5\%。但是现在也有稍稍上升的趋势。相较于国外游,德国人在国内旅游时明显更常使用公车和铁路。



\paragraph{最受喜爱的度假形式\%旅行开销:}

为了能得到尽可能好的休息,德国人特别喜欢在沙滩度假,沙滩之旅以46\%的比例毫无争议拔得头筹。休闲之旅和自然之旅同样也受到他们青睐,分别占比37\%和28\%,差距不大。原则上来说,德国人最爱家庭出游,约有1/4的德国人喜欢和家人一块享受假日。同时探险之旅也很受欢迎,占比24\%,紧随沙滩之旅其后。相反,观光旅行不太受德国人喜爱,仅占比18\%。同样情况的还有徒步旅行,占比17\%。

休假是让人放松的一次奢侈享受,所以德国人用于度假的花费较高。2015年的调查数据显示,如果德国人计划一个5天或更长时间的旅行,平均每人每次旅行的开销为954欧。这一数据较2014年的906欧而言有所升高。平均来看,德国人一次旅行的时间约为12.6天,和2000年相比少了一天。就短期度假旅行而言,德国人一般计划为2到4天,平均每人花销247欧。

\subsubsection{总而言之}

相比于中国人,德国人在旅游有方面更多的追求自由、灵活和探险精神。不同于中国的旅行团的概念,德国人更多的时候以家庭或个人为单位出游。
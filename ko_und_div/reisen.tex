\subsection{中国人旅游}

随着旅游业的日渐兴起,无论过不过节,在世界各地,总能看到中国人的旅游大军,分布在世界的各个角落,网上关于中国旅游的话题也是吐槽多多。那么中国人和外国人的旅游到底有什么区别呢?

中国人旅游喜欢扎堆,特别是在节假日出游,到处都是人挤人,尤其在比较有名的旅游景点,你看到的到处都是乌央乌央的人头,毫无放松旅游的感觉。就算是在国外,比如在泰国游玩拍个照的时候,看到的都是一半人和一半景。但是外敷偶人旅游,同样是在泰国,却可能完全不一样,他们大都似电影里演的那样,很休闲的躺在沙滩椅上,晒着日光浴,而旁边还放一杯鸡尾酒。

扎起中国,许多大叔大妈,爷爷奶奶旅游时都强调低价,什么叫做低价。比如说,“999”玩遍泰国,“598港澳游,全程星级酒店”的这类的低价团,都备受追捧。但实际上就算是很正规的旅游团都会存在强迫购物的项目,更不要提这种明眼人一看便知的购物团了。不过在国外旅游,完全没有低价团这件事,往往都需要付足了各种费用,在泰国和越南这种收小费的习惯,其实就是他们养出来的。

中国人旅游爱逛商业街,国内外各地景色不同,但在中国旅游人的眼中,不到商业街就不算是来到了哪里旅游,尤其是以稍老一辈人最为明显,他们不论去哪儿旅游,往往都会在当地的商场里逛一圈,不论是衣服也好,纪念品也好,都很感兴趣。

中国人旅游在国外人看来,都是不可思议的,在国内人看来也是啼笑皆非。不过,这样的旅游方式其实也是非常有意思的。我觉得,作为年轻人,年轻的时候,在选择旅游方式时可以追求点个性。但是等步入了老年,很多人一起热闹旅游也未必不是好事。

近年来,越来越多的中国人都开始讨厌“扎堆”旅游,假日前夕,总有不少人在网上抱怨未来的人流高峰,甚至发誓绝不出门。但实际上,每逢节假日,国内各个主要景点依旧人山人海,人流量永远有增无减,和网上的舆论形成鲜明对比。为什么都抱怨人多,但中国人还是喜欢到人多的地方旅游呢?这是因为从全局考虑,人多也有人多的好处。
\begin{enumerate}
    
\item \textbf{人多的地方更安全}

许多年轻的旅行者,为了规避人流高峰,喜欢挑选一些人迹罕至的地方去旅游。而人迹罕至,也就意味着这个地方充满了不可知的危险,一旦遇险,你连求援的对象都没有,只能自求多福。而人多的地方,恰恰没有这样的风险。一般来讲,热门景点都已经遭到了完全的开发,排除了大部分未知的危险因素。即便你在这些地方遭到困境,也随时可以向管理人员或身边的游客求助。过去,网上经常有驴友登山失联的消息。试想一下,如果这些山上面到处都是游客,你还会担心失联吗?事实上,在人多的地方,你想要失联都失联不了,因为没走几步,肯定会被人群挤回来。你可能会说,人多难道不是会有踩踏的危险吗?没错,但从统计数据来看,人多地方发生踩踏的概率,远远低于在荒郊野岭被野兽吃掉的概率。再比如说你的手机没电了,在人多的地方就可以很方便的借到充电宝,也是非常幸福的呀。

\item \textbf{人多的地方更温暖}

通常情况下,旅游景点的温度,都低于我们平常生活环境的温度。特别是国庆这样的节日,刚好遇到秋季,气温急剧转凉,大家从温暖的室内出来,奔赴山水之间,最大的感受就是真TMD冷。这个时候,如果你去的碰巧是荒无人烟的地方,那么很可能夜里会在帐篷里冻死。 相反,如果你去的是人流密集的场所,几万人身上散发的热量,很快就会把整个地方熏得热火朝天。

\item \textbf{人多的地方更有趣}

你去那些人群稀少的地方旅游,乍看好像无人打扰,自由自在,但很快你就会陷入空虚和无聊之中。事实上,风景永远是千遍一律的,珠峰上的景色再壮丽,让你在上面独居三十天,是个人都会崩溃。但有一种东西,永远花样翻新,永远不会腻味。没错,那就是人。正所谓“有人的地方就有江湖。”有多少个人,就有多少个江湖。而人多的经典,就好像一个浩瀚的武侠世界。在那些地方,你不仅可以看到美景,而且可以看到身边频频上演的江湖大戏,像情侣吵架、兄弟相争之类的剧目,在景点里随处可见。当然现在最流行的剧目,则是低价旅行团里导游和团客的斗争,揉和了爱情、动作、政治、商战和反乌托邦元素,倍受观众欢迎。去人多的地方旅行,“既是自然之旅,也是人文之旅。”
\item \textbf{主要是从众心理作崽}

中国人素来有凑热闹的传统,只是名气大的地方总想也去打卡一番,即使回来不停吐槽也在所不惜。这类游客通常只是喜欢出去玩,但对旅行的认知并没有到很深的层面,除了被其他人走烂的景点,别的地方都没有什么印象。
\end{enumerate}
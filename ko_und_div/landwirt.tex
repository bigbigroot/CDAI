\section{中国的传统经济背景}

\subsection{地理环境}

常言道:“知者乐水,仁者乐山。知者动,仁者静。知者乐,仁者寿。” 在中国古代认知中世界就是天圆地方。因为他们认识的只是他们生活的这片土地。虽然中国有着漫长的海岸线,但是很少有人会乘船在海上冒险。在中国古代的各种著作中,谈及海洋时,作者只是赞叹海洋的浩瀚广阔。

相比之下德国的情况就稍有不同了。虽然德国仍有许多地方远离海洋,但就距离而言,和中国的许多地方相比还是比较近的。因此,在比重上德国靠海为生的人更多一些。更关键的是德国在文化上也受到了其它海洋国家的影响,相对而言海洋的味道更浓厚一些。

在德国以及欧洲海岸线犬牙交错,大量的海岛渐渐形成了独立的文化。中国在青藏高原以东的地区的山脉则不是那样难以逾越。中国的中心地带从东到西都有可通航的水系连接起来,而东西则有京杭大运河连接。因此,中国于公元前221年完成统一,就再没有长期分裂过。但欧洲的统一都使欧洲的许多伟人折戟沉沙。

\subsection{中国农业为主的经济环境}

中国历来依靠农业生产来维持生存。在中国历史上,战国时期是中国完成统一的前夕。从那个时期开始每一个封建的小国都把“耕战之术”列为国家要务。直到今日,中国也有近一半的人从事农业生产。在一个以农业为主的国家,最重要的就是土地。对于一个中国人甚至是商人,都是用土地的多寡来体现个人以及家族的财富。因此,在中国历史上,一切社会、经济思想以致政策法律都以土地的分配和利用为核心。


在中国的传统思想中一直有本末之分。对于国家来说,农业生产被认为是立国之本,而商业则是一国之末。因为农业有丰富的产出,而商业只是涉及商品交换,对一个国家物产增加关联较少。
而对于个人来说,读书做官,辅佐君王为本,从事商业活动为末。并且中国人一直推崇“耕读传家”。一般人出身在这样的家庭都会引以为傲。

因此,整个社会中的知识阶层,思想意识上都偏向于农业。对于许多中国古代的官员而言,农业生产也和自己家庭的经济状况相关。这导致中国历史上,各种社会、经济理论以及政策多偏向农业。

\subsection{重农主义的形成}

前面讲到中国农业对于国家的重要性。重农主义也就是我们常说的重农抑商的思想。在中国这一思想在春秋战国时期就已经开始形成了。

\subsubsection{上农}

在中国的吕氏春秋中有一篇叫《上农》的文章,其中比较农民和商人的生活方式。其中认为农民像单纯朴实,惯于顺服,他们的财产以土地为主难以移动。因此国家有难时,农民弃置不顾。商人则奸诈自私,计谋多,他们财产简单,易于转移,因此国家有难是,商人往往自己逃跑,不顾国家。在文中,农业不仅相比于商业对国家更重要,农民的生活方式也更为高尚。

吕氏春秋,反映出中国在春秋战国时期诸子百家,虽然各家思想各有不同,但他们都推崇农民、重视农业。

\subsubsection{汉朝商人地位下降}

汉朝作为中国历史上第一个长期统一的朝代,在之前秦朝的基础上确定了许多中国后来两千年来一直沿用的制度政策。重农抑商就是其中之一。

汉朝创立者高帝命令商人“不得衣丝乘车,重租税以困辱之。”,之后因为天下初得安定,重又放宽对商人的法律,然而商人子孙仍不许当官作吏。但渐渐的,富商大贾蓄积财物,奴役贫民;前呼后拥,车乘百余辆;屯积居奇,地方的诸侯对他们也都伏首低眉,仰仗他们供给物资。有的冶铸煮盐,家财积累到万金,而不帮助国家的急难,黎民百姓陷于重困之中。之后有人就说,商人不从事生产,却囤积居奇,财富越来越多,导致越来越多的人从事商业活动,导致粮食产出的减少。粮食价格上涨,政府发行的货币越来越便宜。


\section{中国的传统经济背景}

\subsection{地理环境}

常言道:“知者乐水,仁者乐山。知者动,仁者静。知者乐,仁者寿。” 在中国古代认知中世界就是天圆地方。因为他们认识的只是他们生活的这片土地。虽然中国有着漫长的海岸线,但是很少有人会乘船在海上冒险。在中国古代的各种著作中,谈及海洋时,作者只是赞叹海洋的浩瀚广阔。

相比之下德国的情况就稍有不同了。虽然德国仍有许多地方远离海洋,但就距离而言,和中国的许多地方相比还是比较近的。因此,在比重上德国靠海为生的人更多一些。更关键的是德国在文化上也受到了其它海洋国家的影响,相对而言海洋的味道更浓厚一些。

在德国以及欧洲海岸线犬牙交错,大量的海岛渐渐形成了独立的文化。中国在青藏高原以东的地区的山脉则不是那样难以逾越。中国的中心地带从东到西都有可通航的水系连接起来,而东西则有京杭大运河连接。因此,中国于公元前221年完成统一,就再没有长期分裂过。但欧洲的统一都使欧洲的许多伟人折戟沉沙。
\subsection{重农主义}
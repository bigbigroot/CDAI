\subsection{中国的集体主义精神——舍小家,为大家}
\par
新中国成立以来,中国人民为国家建设都有一种舍小家为大家顽强战斗奋力拼搏的精神,留下了许多先进的事迹。最为大家知道的如铁人王进喜,人民公仆焦裕禄和为人民服务的雷锋等。

\paragraph{铁人王进喜:}
是新中国第一批石油钻探工人,全国著名的劳动模范。他率领钻井队艰苦创业,发扬舍小家顾大家精神,打出来大庆第一块油井,并创造了年进尺10万米的世界钻井记录,为我国石油事业立下汗马功劳。
\paragraph{人民公仆焦裕禄:}
他坚持实事求是、群众路线的领导工作方法,同兰考县全县干部和群众一起,与深重的自然灾害进行顽强斗争,努力改变兰考面貌。他身患肝癌,依旧忍着剧痛,坚持工作,被誉为“党的好干部”、“人民的好公仆”。1964年5月14日,焦裕禄被肝癌夺去了生命,年仅42岁。焦裕禄用自己的实际行动,铸就了亲民爱民、艰苦奋斗、迎难而上、科学求实、无私奉献的焦裕禄精神。
\par
在中国,这些人物事迹往往作为道德楷模,存在于官方的宣传中。这种行为在中国被认为是一种更高的人生境界,即放弃个人利益,为了实现集体或是全人类的目标。
%在日常生活中,我们经常可以看见这些宣传标语。这作为一种鼓励我们奋发图强的精神激励着我们努力工作。
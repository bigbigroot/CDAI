\subsection{职业选择}

专业和职业的选择通常由两个因素影响,一个是自身因素,一个是外界因素。对于自身因素,专业方面和职业上一方面是自身的能力,一方面是个人的兴趣爱好。外界因素上一方面是父母对子女的期望,一方面是社会对人产生的影响。

首先是专业选择,在我们的高中时期,大多数的学生在学习期间并没有也不会考虑对于自己以后大学的专业选择,通常是高考成绩出来,确定了分数,确定了大致的学校,再考虑专业的事情。但是大多数的人并没有什么所谓的对什么专业感兴趣这种概念,没有特定兴趣的,多是与父母交换了一下意见,再在网上或是亲戚处咨询一下选择了就业前景好的热门专业。而有特定兴趣的要是就业前景好也就与上面一样,要是是个冷门专业,可能就会在父母或者亲戚的劝说下改变自己的想法,也有少数人会坚持自己的选择。还有一部分人可能是高考分数到达了目标学校的分数,但是没有到达自己所期望的专业的分数,而又勾选了服从调剂,因此被分到冷门专业。        职业选择与专业选择也是相似。刚到了毕业的时候,大多数的学生,无论是自己还是家人,自然是希望选择待遇好,条件舒适的工作。但是这时和选择专业不同的地方就是能力的大小。专业选择时虽说也有能力的区别,但是总归有大量的与你分数相近的学习与你选择。而选择职业不一样,公司看中你的能力与学历,有能力自然是你挑公司,没有能力,大公司看不上你,小公司又没有前途甚至也可能不要你,很可能就会选择与大学专业无关的工作,可以说是涝的涝死旱的旱死。
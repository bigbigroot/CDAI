\section{大陆国家和海洋国家}
%section 汉萨同盟
\subsection{汉莎同盟}

\subsection{商业与农业}

欧洲的西面与南面被辽阔的大西洋和地中海包围,长长的海岸线使得各地区通过航海进行交流、商贸成为可能。
德国位于欧洲中部,作为东西方之间以及斯堪的纳维亚和地中海地区的中转站,商业贸易活动也十分频繁。在经商的过程中,
商人们不免要时常与数字、账簿打交道,账目上的一串串数字,代表的是金钱、货物的流入流出。因此,在西方,数学和数理推理很早就
作为单独的学科被学者研究,并在商业往来的过程中被广泛的应用。这也是为什么他们的理论结构和语言如此清晰的原因。频繁的商贸活动形成了
以城邦为核心的利益共同体,所以他们的社会组织形式,是以城市为共同利益为基础的。故而,社会组织不是独裁的,因为在一个市民阶级里,没有任何
道德上的理由认为某人应当比别人更重要。在这么一个商业繁荣,各民族往来频繁的环境下,养成了人们善于变化,善于创新的性格特质。

而在过去,中国是一个以农业为核心的农业大国,大部分人都是生活在乡村的农民,它们一年四季,日复一日与自然、土地做斗争,
为了庄稼的收成,他们必须熟谙节气的变化,土地的性质,农作物的特点。因此,农业的研究一致直中国农民、官吏和学者的研究重点,在先秦时代开始,
中国就出现了农家学派和农书,流传至今的《吕氏春秋·上农》就是他们的代表;到北魏时期,出现了中国历史上最伟大的农学著作,被称为"中国古代百科全书"的《齐民要术》。
以农业为核心的产业结构,诞生了以乡村、家庭为单位的社会组织结构,在这样的组织结构中,人们以家族的共同利益为基础,日常的活动都围绕着家族而展开,因此,家长的权力至高无上。
在这么一个小农经济的环境下,人们安土重迁,不轻易做改变,养成了人们顺于自然,轻视人力的性格特质,这也解释了在中国,没有发生类似于工业革命这样的生产力变革。




\subsection{集权与分权——统一与分裂}

德国历史与中国历史一个很大的区别就是德国长期是由许许多多的小王国或是像汉萨同盟这样的城市组成,他们往往具有独立的军队,政府,相互之前没有隶属关系。中国分裂的时期是短暂的,大部分的时间是在一个集权的中央政府的统治之下。

德国是典型的联邦制国家,国家整体与组成部分之间是一种联盟关系,联邦政府行使国家主权、国家最高权力,领导其联邦成员,是对外交往的主体。
联邦各成员都有自己立法和行政机关,尊从代表国家利益的统一法律——联邦宪法。值得注意的是,德国这种现代的联邦制国家历史并不长,也就两百多年,那么其中有什么历史渊源,使这种现代联邦共和制得以诞生呢?
原来,联邦制就起源于过去的小共和国为增强政治、经济、军事力量和谋求共同发展的深切需要而组建的共同体。可以看出,古时德国这种联邦制既是统一的整体,又是分散的部分,因为中央不直接对地方行使政治权力。

而在中国的历史上,不断重复着分分合合的政权交替,大一统的趋势是非常明显的。在周朝,帝王为后裔以及有功之臣分封土地,让他们在地方作为“诸侯”,分区管理,辅佐周王,在诸侯领地内皇帝没有直接的权力,诸侯只需要按时缴纳贡物即可。
这样就造成了诸侯在实力壮大后,诸侯割据,进而反抗周王朝的局面。秦始皇在统一六国以后,吸取了周王朝的教训,着手建立和健全专制主义的中央集权制度,以巩固对全国的统治。在政治层面,皇帝是最高统治者,掌握中央和地方的实权;经济上,
统一货币,度量衡,采取了一系列措施,如“修驰道,车同轨,书同文”,来巩固秦王朝统治;在思想上,规定小篆为全国统一文字,“焚书坑儒”加强思想统治;军事上,修筑万里长城,抵御匈奴南下,派军驻扎东南,岭南。之后,虽然秦王朝二代而亡,但
中央集权的制度一直延续到之后各个封建王朝,中国大一统的趋势是历史潮流。

从以上不难看出,中国各地区之间思想,文化上认同感较之欧洲更强,而在德国这种联邦体制的国家中,各地区间文化上差异更多,如在传统节日上,中国各地区节日基本一致,而德国地方性节日很多,这大概也佐证了中德两国集权与分权的特点。



